\documentclass{article}
\usepackage{amssymb}
\usepackage{amsmath}

\title{Statistical Inference Chapter 1}
\author{Gallant Tsao}

\begin{document}

\maketitle

\begin{enumerate}
    % Question 1
    \item
    \begin{enumerate}
        % 1a
        \item Let $g(x)=x^3$. Then $g$ is monotonically increasing on $(0, 1)$. We get
        \[ 
        g^{-1}(y) = y^{1/3} \implies \frac{d}{dy}g^{-1}(y) = \frac{1}{3y^{2/3}}. 
        \]
        Since $X \in (0, 1), \ Y = X^3 \in (0, 1)$. Then by Theorem 2.1.5, 
        \begin{align*}
        f_{Y}(y) 
            &= f_{X}(g^{-1}(y)) \Bigl|\frac{d}{dy}g^{-1}(y)\Bigr| \\
            &= 42(y^{1/3})^{5}(1 - y^{1/3}) \cdot \frac{1}{3y^{2/3}} \\
            &= 14y(1 - y^{1/3}), \ y \in (0, 1).
        \end{align*}
        We also have 
        \begin{align*}
            \int_{0}^{1} 14y(1 - y^{1/3}) \ dy 
            &= 14\int_{0}^{1} y - y^{4/3} \ dy \\
            &= 14 \Big[ \frac{1}{2}y^2 - \frac{3}{7}y^{7/3} \Big]_{0}^{1} \\
            &= 14 ( \frac{1}{2} - \frac{3}{7}) \\
            &= 1.
        \end{align*}

        % 1b
        \item Let $g(x) = 4x + 3$. Then $g$ is monotonically increasing on $(0, \infty)$. We get 
        \[
        g^{-1}(y) = \frac{y - 3}{4} \implies \frac{d}{dy}g^{-1}(y) = \frac{1}{4}.
        \]
        Since $X \in  (0, \infty), \ Y = 4X + 3 \in (3, \infty)$. Then by Theorem 2.1.5,
        \begin{align*}
            f_{Y}(y)
            &= f_{X}(g^{-1}(y)) \Bigl|\frac{d}{dy}g^{-1}(y)\Bigr| \\
            &= 7e^{-7 \cdot \frac{y - 3}{4}} \cdot \frac{1}{4} \\
            &= \frac{7}{4}e^{\frac{21}{4} - \frac{7}{4}y}, \ y \in (3, \infty).
        \end{align*}
        We also have 
        \begin{align*}
            \int_{3}^{\infty} \frac{7}{4}e^{\frac{21}{4} - \frac{7}{4}y} \ dy 
            &= \frac{7}{4}e^{\frac{21}{4}} \int_{3}^{\infty} e^{-\frac{7}{4}y} \ dy \\
            &= \frac{7}{4}e^{\frac{21}{4}} \Big[ -\frac{4}{7}e^{-\frac{7}{4}y} \Big]_{3}^{\infty} \\
            &= \frac{7}{4}e^{\frac{21}{4}} (\frac{4}{7}e^{-\frac{21}{4}}) \\
            &= 1.
        \end{align*}

        % 1c
        \item Let $g(x) = x^2$. Then $g$ is monotonically increasing on $(0, 1)$. We get 
        \[
        g^{-1}(y) = \sqrt{y} \implies \frac{d}{dy}g^{-1}(y) = \frac{1}{2\sqrt{y}}.
        \]
        Since $X \in (0, 1), \ Y = X^2 \in (0, 1)$. Then by Theorem 2.1.5, 
        \begin{align*}
            f_{Y}(y) 
            &= f_{X}(g^{-1}(y)) \Bigl|\frac{d}{dy}g^{-1}(y)\Bigr| \\
            &= 30y(1 - \sqrt{y})^2 \cdot \frac{1}{2\sqrt{y}} \\
            &= 15\sqrt{y}(1 - \sqrt{y})^2, \ y \in (0, 1).
        \end{align*}
        We also have 
        \begin{align*}
            \int_{0}^{1} 15\sqrt{y}(1 - \sqrt{y})^2 \ dy
            &= 15 \int_{0}^{1} \sqrt{y} - 2y + y^{3/2} \ dy \\
            &= 15 \Big[ \frac{2}{3}y^{3/2} - y^2 + \frac{2}{5}y^{5/2} \Big]_{0}^{1} \\
            &= 15(\frac{2}{3} - 1 + \frac{2}{5}) \\
            &= 1.
        \end{align*}
        
    \end{enumerate}

    % Question 2
    \item 
    \begin{enumerate}
        \item Let $g(x) = x^2$. Then $g$ is monotonically increasing on $(0, 1)$. We get 
        \[
        g^{-1}(y) = \sqrt{y} \implies \frac{d}{dy}g^{-1}(y) = \frac{1}{2\sqrt{y}}.
        \]
        Since $X \in (0, 1), \ Y = X^2 \in (0, 1)$. Then by Theorem 2.1.5,
        \begin{align*}
            f_{Y}(y)
            &= f_{X}(g^{-1}(y)) \Bigl|\frac{d}{dy}g^{-1}(y)\Bigr| \\
            &= 1 \cdot \frac{1}{2\sqrt{y}} \\
            &= \frac{1}{2\sqrt{y}}, \ y \in (0, 1).
        \end{align*}

        \item Let $g(x) = -\log{x}$. Then $g$ is monotonically decreasing on $(0, 1)$. We get 
        \[
        g^{-1}(y) = e^{-y} \implies \frac{d}{dy}g^{-1}(y) = -e^{-y}.
        \]
        Since $X \in (0, 1), \ Y = \log{X} \in (0, \infty)$. Then by Theorem 2.1.5, 
        \begin{align*}
            f_{Y}(y)
            &= f_{X}(g^{-1}(y)) \Bigl|\frac{d}{dy}g^{-1}(y)\Bigr| \\
            &= \frac{(n + m + 1)!}{n!m!}e^{-ny}(1 - e^{-y})^{m} \cdot |-e^{-y}| \\
            &=  \frac{(n + m + 1)!}{n!m!}e^{-y(n + 1)}(1 - e^{-y})^{m}, \ y \in (0, \infty).
        \end{align*}

        \item Let $g(x) = e^{x}$. Then $g$ is monotonically increasing on $(0, \infty)$. We get 
        \[
        g^{-1}(y) = \ln{y} \implies \frac{d}{dy}g^{-1}(y) = \frac{1}{y}.
        \]
        Since $X \in (0, \infty), \ Y = e^{X} \in (0, \infty)$. Then by Theorem 2.1.5, 
        \begin{align*}
            f_{Y}(y)
            &= f_{X}(g^{-1}(y)) \Bigl|\frac{d}{dy}g^{-1}(y)\Bigr| \\
            &= \frac{1}{\sigma^2} \ln{y} e^{-(\ln{y}/\sigma)^2 /2} \cdot \frac{1}{y} \\
            &= \frac{1}{\sigma^2} \frac{\ln{y}}{y} e^{-(\ln{y}/\sigma)^2 / 2}, \ y \in (0, \infty).
        \end{align*}
    \end{enumerate}

    % Question 3
    \item First of all, 
    \[ X \in \{0, 1, 2, ...\} \implies Y \in \Big\{0, \frac{1}{2}, \frac{2}{3}, ...\Big\}. \]
    Then
    \begin{align*}
        P(Y = y) 
        &= P(\frac{X}{X + 1} = y) \\
        &= P(1 - \frac{1}{X + 1} = y) \\
        &= P(X = \frac{y}{1 - y}) \\
        &= \frac{1}{3} \Big( \frac{2}{3} \Big)^{y / (1 - y)}, \ y \in \Big\{\frac{k}{k + 1}: k \in \mathbb{N}_0 \Big\}.
    \end{align*}
    

\end{enumerate}

\end{document}