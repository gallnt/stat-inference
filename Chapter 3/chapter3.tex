\documentclass{article}
\usepackage{amssymb}
\usepackage{amsmath}
\usepackage{graphicx}
\usepackage[margin=1.5in]{geometry}

\title{Statistical Inference Chapter 3}
\author{Gallant Tsao}

\begin{document}

\maketitle

\begin{enumerate}
    % 3.1
    \item We first note that the pmf of $X$ is 
    \[ p_X(x) = \frac{1}{N_1 - N_0 + 1}, \ x \in \{N_0, N_0 + 1, ..., N_1\}. \]
    Then we get the expectation to be 
    \begin{align*}
        \mathbb{E}[X]
        &= \sum_{x = N_0}^{N_1} x\frac{1}{N_1 - N_0 + 1} \\
        &= \frac{1}{N_1 - N_0 + 1} \cdot \frac{N_1 - N_0 + 1}{2} (2N_0 + (N_1 - N_0 + 1 - 1)) \\
        &= \frac{N_1 + N_0}{2}.
    \end{align*}
    As for the variance, we get 
    \begin{align*}
        \mathbb{E}[X^2]
        &= \sum_{x = N_0}^{N_1} x^2 \frac{1}{N_1 - N_0 + 1} \\ 
        &= \frac{1}{N_1 - N_0 + 1} \Big( \sum_{x = 1}^{N_1}x^2 - \sum_{x = 1}^{N_0 - 1} x^2 \Big) \\
        &= \frac{1}{N_1 - N_0 + 1} \Big( \frac{N_1(N_1 + 1)(N_1 + 2) - (N_0 - 1)(N_0)(2N_0 - 1)}{6} \Big) \\
    \end{align*}
    So that 
    \begin{align*}
        Var(X)
        &= \mathbb{E}[X^2] - (\mathbb{E}[X])^2 \\
        &= 1
    \end{align*}

    % 3.2
    \item Let $X = $ number of defective parts in the sample. Then \\
    $X \sim \text{Hypergeometric}(100, n, K)$.
    \begin{enumerate}
        \item Firstly, we need $n = 6$ because for the same $K$, increasing $n$ decreases the 
        value of the Hypergeometric pmf (image shown at end of answer). Then with $n = 6$, 
        \begin{align*}
            P(X = 0) 
            &= \frac{\binom{6}{0}\binom{94}{K}}{\binom{100}{K}} \\
            &= \frac{(100 - k)\cdots (100 - K - 5)}{100 \cdots 95}.
        \end{align*}
        After some trial and error with the calculations, we have that when 
        $K = 31$, $P(X = 0) = 0.10056$, but when $K = 32$, $P(X = 0) = 0.09182$. Therefore, 
        the sample size must be at least 32.
        \begin{center}
            \includegraphics*[width=0.8\textwidth]{../scripts/3-2.png}
        \end{center}
        
        \item By the same reasoning above, we need $n = 6$. Then with this $n$,
        \begin{align*}
            P(X = 0 \text{ or } 1) 
            &= \frac{\binom{6}{0}\binom{94}{K}}{\binom{100}{K}} 
            + \frac{\binom{6}{1}\binom{94}{K - 1}}{\binom{100}{K}}.
        \end{align*}
        Again, by trial and error, when $K = 50$, $P(X = 0 \text{ or } 1) = 0.10220$, but when 
        $K = 51$, $P(X = 0 \text{ or } 1) =0.09331$ hence the sample size must be at least 51.
    \end{enumerate}

    % 3.3
    \item During the three seconds that the person is crossing, there should be no cars passing. 
    The probability of this happening is $(1 - p)^3$. The only possibility for the person to 
    not wait exactly 4 seconds is when there is a car at the first second and no cars in the 
    next 3 seconds. The probability of this happening is $p(1 - p)^3$. Since the times are 
    independent, the probability that the pedestrian has to wait exactly 4 seconds is 
    $[1 - p(1 - p)^3] (1 - p)^3$.

    % 3.4
    \item \begin{enumerate}
        \item Let $X$ be the number of trials. Then in this case $X \sim \text{Geom}(0.1)$. 
        Therefore the mean number of trials is just $\frac{1}{0.1} = 10$.

        \item 
    \end{enumerate}

    % 3.5
    \item Let $X = $ number of effective cases. Suppose the new drug is equally effective as 
    the old drug. Then $X \sim \text{Binomial}(100, 0.8)$ if the cases are independent from each 
    other, which is a reasonable assumption. We have 
    \[ P(X \geq 85) = \sum_{k = 85}^{100}\binom{100}{k} 0.8^{k} \cdot 0.2^{100 - k} = 0.1285. \]
    From this, the probability of getting 85 or more effective cases is not too small, hence we 
    cannot directly make a conclusion that the new drug is superior.

    % 3.6
    \item \begin{enumerate}
        \item $X \sim \text{Binomial}(2000, 0.01)$.

        \item \[ \sum_{k = 0}^{99} \binom{2000}{k} 0.01^{k} \cdot 0.99^{2000 - k}. \]

        \item In our problem, $n = 2000, p = 0.01, q = 0.99$. Since $np, nq > 5$, we can use 
        normal approximation here. The normal approximation is $Y \sim N(\mu, \sigma^2)$, where 
        \[ \mu = np = 20, \sigma^2 = npq = 19.8. \]
        Then we get 
        \[ P(X < 100) \approx P(Z < 17.979) = 1. \]
    \end{enumerate}

    % 3.7
    \item Let $X$ be the number of chocolate chips in the cookie. Then 
    $X \sim \text{Poisson}(\lambda)$. We want that 
    \[ P(X \geq 2) = 1 - P(X \leq 1) > 0.99 \implies P(X \leq 1) 
    = e^{-\lambda} + \lambda e^{-\lambda} < 0.01. \]
    Solving the above numerically, we get that $\lambda = 6.6384.$

    % 3.8
    \item \begin{enumerate}
        \item Let $X$ be the number of customers in the theater. Then 
        $X \sim \text{Binomial}(1000, \frac{1}{2}).$ We want 
        \[ P(X > N) = \sum_{k = N + 1}^{1000} \binom{1000}{k} \Big( \frac{1}{2} \Big)^{k} 
        \Big( 1 - \frac{1}{2} \Big)^{1000 - k} < 0.01.\]
        In other words, we are solving the smallest $N$ such that 
        \[ \Big( \frac{1}{2} \Big)^{1000} \sum_{k = N + 1}^{1000} \binom{1000}{k} < 0.01. \]
        By looping over $N$, we eventually get that $N = 537$.

        \item $n = 1000, p = q = \frac{1}{2}$. Therefore the parameters for the normal 
        approximation are $\mu = np = 500, \sigma^2 = npq = 250$. Then we are solving for 
        \[ P(X > N) \approx P(Z > \frac{N - 500}{\sqrt{250}}) < 0.01. \]
        Using R, we get that 
        \[ \frac{N - 500}{\sqrt{250}} = 2.326 \implies N \approx 537, \]
        which is the same as our answer in part (a).
    \end{enumerate} 

    % 3.9
    \item \begin{enumerate}
        \item Let $X \sim \text{Binomial}$ as depicted in the question. 
        \begin{align*}
            P(X \geq 5)
            &= 1 - P(X \leq 4) \\
            &= 1 - \sum_{k = 0}^{4} \binom{60}{k} \Big( \frac{1}{90} \Big)^{k} 
            \Bigl( 1 - \frac{1}{90} \Bigr)^{60 - k} \\
            &\approx 0.0006,
        \end{align*}
        which I think is rare enough to be on the news.

        \item Let $X$ be the number of schools in New York state with 5 or more sets of twins. Then 
        $X \sim \text{Binomial}(360, 0.0006)$. We have that 
        \[ P(X \geq 1) = 1 - P(X = 0) \approx 0.1698. \]

        \item Let $X$ be the number of states in the past 10 years having 5 or more sets of twins. Then 
        $X \sim \text{Binomial}(500, 0.1698)$. We have that 
        \[ P(X \geq 1) = 1 - P(X = 0) = 1. \]
        Therefore this event becomes almost certain as we broaden the time scope.
    \end{enumerate}

    % 3.10
\end{enumerate}

\end{document}