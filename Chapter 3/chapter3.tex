\documentclass{article}
\usepackage{amssymb}
\usepackage{amsmath}
\usepackage{graphicx}
\usepackage[margin=1.5in]{geometry}

\title{Statistical Inference Chapter 3}
\author{Gallant Tsao}

\begin{document}

\maketitle

\begin{enumerate}
    % 3.1
    \item We first note that the pmf of $X$ is 
    \[ p_X(x) = \frac{1}{N_1 - N_0 + 1}, \ x \in \{N_0, N_0 + 1, ..., N_1\}. \]
    Then we get the expectation to be 
    \begin{align*}
        \mathbb{E}[X]
        &= \sum_{x = N_0}^{N_1} x\frac{1}{N_1 - N_0 + 1} \\
        &= \frac{1}{N_1 - N_0 + 1} \cdot \frac{N_1 - N_0 + 1}{2} (2N_0 + (N_1 - N_0 + 1 - 1)) \\
        &= \frac{N_1 + N_0}{2}.
    \end{align*}
    As for the variance, we get 
    \begin{align*}
        \mathbb{E}[X^2]
        &= \sum_{x = N_0}^{N_1} x^2 \frac{1}{N_1 - N_0 + 1} \\ 
        &= \frac{1}{N_1 - N_0 + 1} \Big( \sum_{x = 1}^{N_1}x^2 - \sum_{x = 1}^{N_0 - 1} x^2 \Big) \\
        &= \frac{1}{N_1 - N_0 + 1} \Big( \frac{N_1(N_1 + 1)(N_1 + 2) - (N_0 - 1)(N_0)(2N_0 - 1)}{6} \Big) \\
    \end{align*}
    So that 
    \begin{align*}
        Var(X)
        &= \mathbb{E}[X^2] - (\mathbb{E}[X])^2 \\
        &= 1
    \end{align*}

    % 3.2
    \item Let $X = $ number of defective parts in the sample. Then \\
    $X \sim \text{Hypergeometric}(100, n, K)$.
    \begin{enumerate}
        \item Firstly, we need $n = 6$ because for the same $K$, increasing $n$ decreases the 
        value of the Hypergeometric pmf (image shown at end of answer). Then with $n = 6$, 
        \begin{align*}
            P(X = 0 | n = 6) 
            &= \frac{\binom{6}{0}\binom{94}{K}}{\binom{100}{K}} \\
            &= \frac{(100 - k)\cdots (100 - K - 5)}{100 \cdots 95}
        \end{align*}
        After some trial and error with the calculations, we have that when 
        $K = 31$, $P(X = 0) = 0.10056$, but when $K = 32$, $P(X = 0) = 0.09182$. Therefore, 
        the sample size must be at least 32.
        \begin{center}
            \includegraphics*[width=0.8\textwidth]{../scripts/3-2.png}
        \end{center}
        
        \item By the same reasoning above, we need $n = 6$. Then 
    \end{enumerate}
\end{enumerate}

\end{document}