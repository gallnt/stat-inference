\documentclass{article}
\usepackage{amsmath}
\usepackage{amssymb}
\usepackage[margin=1.5in]{geometry}

\title{Statistical Inference Chapter 1}
\author{Gallant Tsao}

\begin{document}

\maketitle

\begin{enumerate}
    % 1.1
    \item \begin{enumerate}
        \item $\Omega = \{(x_1, x_2, x_3, x_4): x_i \in \{H, T\}\}.$
        \item If there are N leaves on the plant, $\Omega = [N]$.
        \item $\Omega = \{t: t \in \mathbb{R},\ t \geq 0\}$.
        \item $\Omega = \{w: w \in \mathbb{R}_{+}\}$.
        \item If there are n components, $\Omega = \{i/n: i \in \{0, 1, ..., n\}\}$.
    \end{enumerate}

    % 1.2
    \item \begin{enumerate}
        \item \begin{align*}
            x \in A \setminus B 
            &\iff x \in A \text{ and } x \notin B \\
            &\iff x \in A \text{ and } x \notin A \cap B \\
            &\iff x \in A \setminus (A \cap B).
        \end{align*}
        Also, 
        \begin{align*}
            x \in A \setminus B
            &\iff x \in A \text{ and } x \notin B \\
            &\iff x \in A \text{ and } x \in B^{c} \\
            &\iff x \in A \cap B^{c}.
        \end{align*}
        Therefore $A \setminus B = A \setminus (A \cap B) = A \cap B^{c}$.

        \item By the distributive law, \begin{align*}
            (B \cap A) \cup (B \cap A^{c}) 
            &= B \cap (A \cup A^c) \\
            &= B.
        \end{align*}

        \item \begin{align*}
            x \in B \setminus A
            &\iff x \in B \text{ and } x \notin A \\
            &\iff x \in B \text{ and } x \in A^c \\
            &\iff x \in B \cap A^c.
        \end{align*}

        \item From part b), we have \begin{align*}
            A \cup B
            &= A \cup ((B \cap A) \cup (B \cap A^c)) \\
            &= A \cup (B \cap A) \cup A \cup (B \cap A^c) \\
            &= A \cup A \cup (B \cap A^c) \\
            &= A \cup (B \cap A^c).
        \end{align*}
    \end{enumerate}

    % 1.3
    \item \begin{enumerate}
        \item \begin{align*}
            x \in A \cup B 
            &\iff x \in A \text{ or } x \in B \\
            &\iff x \in B \cup A. \\
            x \in A \cap B
            &\iff x \in A \text{ and } x \in B \\
            &\iff x in B \cap A.
        \end{align*}

        \item \begin{align*}
            x \in A \cup (B \cup C) 
            &= x \in A \text{ or } x \in B \cup C \\
            &= x \in A \cup B \text{ or } x \in C \\
            &= x \in (A \cup B) \cup C.
        \end{align*}

        \item \begin{align*}
            x \in (A \cup B)^{c}
            &\iff x \notin A \cup B \\
            &\iff x \in A^c \text{ and } x \in B^c \\
            &\iff x \in A^c \cap B^c. \\
            x \in (A \cap B)^c
            &\iff x \notin A \cap B \\
            &\iff x \in A^c \text{ or } x \in B^c \\
            &\iff x \in A^c \cup B^c.
        \end{align*}
    \end{enumerate}

    % 1.4
    \item \begin{enumerate}
        \item This is $P(A \cup B)$, so we get $P(A) + P(B) - P(A \cap B)$.
        \item This is $P(A \Delta B)$, so we get $P(A) + P(B) - 2P(A \cap B)$.
        \item This is again $P(A \cup B)$, so we get $P(A) + P(B) - P(A \cap B)$.
        \item This is $P((A \cap B)^c)$, so we get $1 - P(A \cap B)$.
    \end{enumerate}

    % 1.5
    \item \begin{enumerate}
        \item $A \cap B \cap C = \{ \text{a U.S. birth resulting in identical twin females} \}$.
        \item $P(A \cap B \cap C) = \frac{1}{90} \cdot \frac{1}{3} \cdot \frac{1}{2} 
        = \frac{1}{540}$.
    \end{enumerate}

    % 1.6
    \item $p_0 = (1 - u)(1 - w), p_1 = u(1 - w) + w(1 - u), p_2 = uw$. For them to be equal, 
    \begin{align*}
        p_0 = p_2 
        &\implies 1 - u - w + uw = uw \\
        &\implies u + w = 1, \\
        p_1 = p_2
        &\implies u + w - 2uw = uw \\
        &\implies uw = \frac{1}{3}.
    \end{align*}
    The above two equations imply $u(1 - u) = \frac{1}{3}$, which has no real solutions in 
    $\mathbb{R}$. Therefore we can't choose such $u, w$ satisfying $p_0 = p_1 = p_2$.
    % 1.7
    \item \begin{enumerate}
        \item This is just having an extra case of hitting outside of the dart board. So 
        \[ P(\text{scoring } i \text{ points}) = \begin{cases}
            1 - \frac{\pi r^2}{A} & i = 0 \\
            \frac{\pi r^2}{A} \cdot \frac{1}{5^2} ((6 - i)^2 - (5 - i)^2) & i = 1, ..., 5
        \end{cases} \]

        \item \begin{align*}
            P(\text{scoring } i \text{ points}|\text{board is hit})
            &= \frac{P(\text{scoring } i \text{ points, board is hit})}
            {P(\text{board is hit})} \\
            &= \frac{\pi r^2}{A} \cdot \frac{1}{5^2} ((6 - i)^2 - (5 - i)^2) 
            / \frac{\pi r^2}{A} \\
            &= \frac{1}{5^2} ((6 - i)^2 - (5 - i)^2), \ i = 1, ..., 5
        \end{align*}
        For $i = 0$, we will definitely score given that we hit the board so \\
        $P(\text{scoring 0 points} | \text{board is hit}) = 0$, which is consistent with the 
        probability distribution in Example 1.2.7 as well.
    \end{enumerate}

    % 1.8
    \item \begin{enumerate}
        \item From the example given, \[
            P(\text{scoring } i \text{ points}) = \frac{(6 - i)^2 - (5 - i)^2}{5^2}, i = 1, ..., 5.
            \]

        \item Expanding the above, 
        \[ \frac{(6 - i)^2 - (5 - i)^2}{5^2} = \frac{11 - 2i}{r^2}, \]
        which is a decreasing function of $i$.

        \item \[
        \frac{11 - 2i}{5^2} > 0 \text{for } i = 1, ..., 5
        \]
        hence the first axiom is satisfied.
        \[ P(S) = P(\text{hitting the board}) = 1, \]
        hence the second axiom is satisfied. For $i \neq j$,
        \[ P(i \cup j) = \text{Area of ring } i + \text{Area of ring } j 
        = P(i) + P(j), \]
        hence the third axiom is satisfied so $P(\text{scoring } i \text{ points})$ 
        is a probability function.
    \end{enumerate}
    
    % 1.9
    \item \begin{enumerate}
        \item Suppose $x \in (\cup_{\alpha} A_{\alpha})^c$. Then $x \notin A_{\alpha}$ for all 
        $\alpha \in \Gamma$ so $x \in A_{\alpha}^{c}$ for all $\alpha \in \Gamma$. Therefore 
        $x \in \cap_{\alpha} A_{\alpha}$.

        Now suppose $x \in \cap_{\alpha} A_{\alpha}^{c}$. Then for all $\alpha \in \Gamma$, 
        $x \in A_{\alpha}^{c}$ hence $x \notin A_{\alpha}$, then 
        $x \notin \cup_{\alpha} A_{\alpha}$ so $x \in (\cup_{\alpha} A_{\alpha})^{c}$.

        \item Suppose $x \in (\cap_{\alpha} A_{\alpha})^{c}$. Then 
        $x \notin \cap_{\alpha} A_{\alpha}$ so $x \notin A_{\alpha}$ for some 
        $\alpha \in \Gamma$. Then $x \in A_{\alpha}^{c}$ for some $\alpha \in \Gamma$. 
        Therefore $x \in \cup_{\alpha} A_{\alpha}^{c}$.

        Now suppose $x \in \cup_{\alpha} A_{\alpha}^{c}$. Then $x \in A_{\alpha}^{c}$ for some 
        $\alpha \in \Gamma$ so $x \notin A_{\alpha}$ for some $\alpha \in \Gamma$. Then 
        $x \notin \cap_{\alpha}$ thus $x \in (\cap_{\alpha})^{c}$.
    \end{enumerate}

    % 1.10
    \item We have \[
    \Big( \bigcup_{i = 1}^{n} A_i \Big)^{c} = \bigcap_{i = 1}^{n} A_{i}^{c}, \
    \Big( \bigcap_{i = 1}^{n} A_i \Big)^{c} = \bigcup_{i = 1}^{n} A_{i}^{c}
    \]
    Proof of first equality: \\
    Suppose $x \in (\cup_{i = 1}^{n} A_i)^c$. Then $x \notin \cup_{i = 1}^{n} A_i$ so 
    $x \notin A_i$ for all $i$, meaning $x \in A_{i}^{c}$ for all $i$. Therefore 
    $x \in \cap_{i = 1}^{n} A_{i}^{c}$. 
    Now suppose $x \in \cap_{i = 1}^{n} A_{i}^{c}$. 
    Then $x \notin A_i$ for all $i$, hence $x \notin \cup_{i = 1}^{n} A_{i}$, hence 
    $x \in (\cup_{i = 1}^{n} A_i)^c$.

    Proof of second equality: \\
    Suppose $x \in (\cap_{i = 1}^{n} A_i)^{c}$. Then $x \notin \cap_{i = 1}^{n} A_i$ so 
    $x \notin A_i$ and so $x \in A_{i}^{c}$ for some $i$, meaning 
    $x \in \cup_{i = 1}^{n} A_i^c$. 
    Now suppose $x \in \cup_{i = 1}^{n} A_i^c$. Then $x \notin A_i$ for some $i$ hence 
    $x \in (\cap_{i = 1}^{n} A_i)^{c}$.

    % 1.11
    \item \begin{enumerate}
        \item $\emptyset \in \mathcal{B}$ hence property 1 is satisfied. 
        $\emptyset^{c} = S \in \mathcal{B}, \ S^{c} = \emptyset \in \mathcal{B}$ hence 
        property 2 is satisfied. $\emptyset \cup S = S \in \mathcal{B}$ hence property 
        3 is satisfied so $\mathcal{B}$ is a sigma algebra.
        
        \item $\emptyset$ is a subset of $S$ hence $\emptyset \in \mathcal{B}$ hence property 
        1 is satisfied. For any set $A \in \mathcal{B}, \ A^{c} = S \setminus A \in B$ hence 
        property 2 is satisfied. Any finite union of elements in $\mathcal{B}$ will be a 
        subset of $S$, which will be in $\mathcal{B}$ so $\mathcal{B}$ is a sigma algebra.

        \item Suppose $\mathcal{F}_1, \mathcal{F}_2$ are sigma algebras on the sample space
        $S$. Since $\emptyset \in \mathcal{F}_1$ and $\emptyset \in \mathcal{F}_2$, 
        $\emptyset \in \mathcal{F}_1 \cap \mathcal{F}_2$ so property 1 is satisfied. 
        Suppose $A \subseteq \mathcal{F}_1 \cap \mathcal{F}_2$. Then $A \subseteq \mathcal{F}_1$ 
        and $A \subseteq \mathcal{F}_2$. Since $\mathcal{F}_1, \mathcal{F}_2$ are both sigma 
        algebras, $A^{c} \in \mathcal{F}_1$ and $A^{c} \in \mathcal{F}_2$. Therefore 
        $A^{c} \in \mathcal{F}_1 \cap \mathcal{F}_2$ so property 2 is satisfied.
        Suppose $A_1, A_2, \dots \in \mathcal{F}_1 \cap \mathcal{F}_2$. Then 
        $A_i \in \mathcal{F}_1$ and $A_i \in \mathcal{F}_2$. Since $\mathcal{F}_1, 
        \mathcal{F}_2$ are both sigma algebras, $\cup_{i} A_i \in \mathcal{F}_1$ and 
        $\cup_{i} A_i \in \mathcal{F}_2$ hence $\cup_{i} A_i \in \mathcal{F}_1 
        \cap \mathcal{F}_2$ hence property 3 is satisfied so $\mathcal{F}_1 \cap \mathcal{F}_2$ 
        is a sigma algebra.
    \end{enumerate}

    % 1.12
    \item \begin{enumerate}
        \item 12.1
    \end{enumerate}

    % 1.13
    \item $A, B$ cannot be disjoint. If they are, \[
        P(A \cup B) = P(A) + P(B) = \frac{1}{3} + \frac{1}{4} = \frac{13}{12} > 1,
    \]
    which is not possible.

    % 1.14
    \item For each element, we can choose to include it or exclude it in the subset. Since there 
    are $n$ elements, the number of subsets that can be formed is $2^{n}$. A more formal proof 
    can be done using bijections.

    % 1.15
    \item Now that the base case of $k = 2$ has been done, assume that this is true for $k$ 
    seperate tasks. Then for each of the $n_1 \times n_2 \times \cdots \times n_k$ ways, we 
    have $n_{k + 1}$ choices for the $(k + 1)$th task. Therefore the entire job can be done in 
    \[ \underbrace{1 \times n_{k + 1} + 1 \times n_{k + 1} + \dots + 1 \times n_{k + 1}}
    _{n_1 \times ... \times n_k \text{ terms}} = n_1 n_2 \cdots n_{k + 1}. \]

    % 1.16
    \item \begin{enumerate}
        \item $26^3$
        \item $26^3 + 26^2$
        \item $26^4 + 26^3 + 26^2$
    \end{enumerate}

    % 1.17
    \item This is just choosing 2 numbers out of $n$ of them, which is $\binom{n}{2} 
    = \frac{n(n + 1)}{2}$.

    % 1.18
    \item There are a total of $n^n$ ways of putting $n$ balls into $n$ cells. For exactly one 
    cell to be empty, there will also be another cell which has exactly 2 balls in it. Therefore 
    there are $\binom{n}{2}$ ways of picking these special buckets. Since the order of putting 
    in the balls matters, the answer is $\binom{n}{2}n! / n^n$.

    % 1.19
    \item \begin{enumerate}
        \item By part (b), this is $\binom{6}{4} = 15$.

        \item We can consider the $n$ variables as bins, and the $r$ partial derivatives as 
        balls. Then we are putting $r$ unlabeled balls into $n$ unlabeled bins. There are a 
        total of $\binom{n + r - 1}{n - 1} = \binom{n + r - 1}{r}$ ways of doing this.    
    \end{enumerate}

    % 1.20
    \item First of all, there are many different ways such that there is at least one call per day. Staying consistent with Casella's answers, if there is 6 calls on 1 day and 1 call on 
    the other six days, we will denote this configuration as 6111111. All possible configs and 
    the number of ways to form them are shown in the table below:
    \[ \begin{tabular}{|l|l|r|}
        \hline
        \text{Config} & \text{Number of Ways} & \text{Answer} \\ 
        \hline
        6111111 & $7\binom{12}{7} \cdot 6!$ & 4656960 \\
        5211111 & $7\binom{12}{5} \cdot 6\binom{7}{2} \cdot 5! $ & 82825280 \\
        4221111 & $7\binom{12}{4} \cdot \binom{6}{2}\binom{8}{2}\binom{6}{2} \cdot 4!$ 
        & 523908000 \\
        4311111 & $7\binom{12}{4} \cdot 6\binom{8}{3} \cdot 5! $ & 139708800 \\
        3321111 & $\binom{7}{2}\binom{12}{3}\binom{9}{3} \cdot 5\binom{6}{2} \cdot 4!$ 
        & 698544000 \\
        3222111 & $7\binom{12}{3} \cdot \binom{6}{3}\binom{9}{2}\binom{7}{2}\binom{5}{2} \cdot 3!$ 
        & 1397088000 \\
        2222211 & $\binom{7}{5}\binom{12}{2}\binom{10}{2}\binom{8}{2}\binom{6}{2}\binom{4}{2}
        \cdot 2!$ & 314344800 \\
        Total & & 3162075840 \\
        \hline
    \end{tabular} \]
    For example, for the config 6111111, there are $\binom{12}{6}$ ways for picking the calls 
    for the day with 6 calls, 7 ways for the 6-call day to be in, and $6!$ ways for rearranging 
    the rest of the 1-call days. A similar reasoning follows for the rest of the configs as well.
    All in all, the answer is about
    \[ \frac{3162075840}{7^{12}} \approx 0.2285. \]

    % 1.21
    \item There are $\binom{2n}{2r}$ ways of choosing the shoes. For there to be no matching pair, 
    there are $\binom{n}{2r}$ ways of choosing, and for each choice within the $2r$ shoes, it can 
    be either a left or right foot so there is a factor of $2^{2r}$. Therefore out final answer is 
    $\binom{n}{2r}2^{2r} / \binom{2n}{2r}$.

    % 1.22
    \item \begin{enumerate}
        \item We need 15 days from each month, hence our answer is 
        \[ \frac{\binom{31}{15}\binom{30}{15}\cdots\binom{31}{15}}{\binom{366}{180}} 
        \approx 0.167 \times 10^{-8}. \]

        \item We can just exclude the days from September so our answer is 
        $\binom{336}{30} / \binom{366}{30}.$
    \end{enumerate}

    % 1.23
    \item There can be 0 to $n$ heads for both players, which are disjoint events. Therefore 
    \begin{align*}
        P(\text{Same number of heads}) 
        &= \Big[ \sum_{x = 0}^{n} \binom{n}{x} \Big(\frac{1}{2} \Big)^{x} 
        \Big(\frac{1}{2} \Big)^{n - x}\Big]^2 \\
        &= \Big( \frac{1}{4} \Big)^{n} \sum_{x = 0}^{n} \binom{n}{x}^{2} \\
        &= \binom{2n}{n} \Big( \frac{1}{4} \Big)^{n}.
    \end{align*}
    (Note that the summation ends up in $\binom{2n}{n}$ as one can think about this being 
    equivalent to choosing $n$ people from $2n$ people: We divide the $2n$ people into two 
    groups of $n$ people. We can pick $k$ people from the first group and pick $n - k$ from 
    the second group. A more formal proof uses generating functions.)

    % 1.24
    \begin{enumerate}
        \item Player A can win on the 1st, 3rd, ..., toss. We have 
        \begin{align*}
            P(\text{A wins}) 
            &= \sum_{k = 1}^{\infty} P(\text{A wins on } k \text{th toss}) \\
            &= \sum_{k = 1}^{\infty} \frac{1}{2} \Big( \frac{1}{2} \Big)^{2k - 2} \\
            &= \frac{2}{3}.
        \end{align*}

        \item With the same idea as above, 
        \begin{align*}
            P(\text{A wins}) 
            &= \sum_{k = 1}^{\infty} P(\text{A wins on } k \text{th toss}) \\
            &= \sum_{k = 1}^{\infty} p (1 - p)^{2k - 2} \\
            &= \frac{p}{1 - (1-p)^2}.
        \end{align*}

        \item Taking the derivative with respect to p, 
        \[ \frac{d}{dp} \frac{p}{1 - (1-p)^2} = \frac{p^2}{(1 - (1 - p)^2)^2} > 0. \]
        Therefore this function is an increasing function in $p$, and its minimum occurs at 
        $p = 0$. By L'Hopital's rule we have 
        \[ \lim_{p \to 0^+} \frac{p}{1 - (1 - p)^2} = \frac{1}{2}, \]
        hence for $p \in (0, 1), \ P(\text{A wins}) > \frac{1}{2}$.
    \end{enumerate}

    % 1.25
    Suppose that the order matters for the two children. Then 
    \begin{align*}
        &\qquad P(\text{Both children are boys | at least one is a boy}) \\
        &= \frac{P(\text{Both children are boys, at least one is a boy})}
        {P(\text{At least one is a boy})} \\
        &= \frac{1}{3}.
    \end{align*}

    % 1.26
    Let $X$ be the number of tosses until a 6 appears. Then $X \sim \text{Geom}(\frac{1}{6})$.
    \begin{align*}
        P(X > 5) 
        &= 1 - P(X \leq 4) \\
        &= 1 - \sum_{k = 0}^{4} \frac{1}{6} \Big( \frac{5}{6} \Big)^{k} \\
        &= ...
    \end{align*}

    % 1.27
    \begin{enumerate}
        \item If $n$ is odd, each $k$ term cancels out with the $n - k$ term so the statement 
        is correct. If $n$ is even, by Pascal's identity, 

        \item By the Binomial Theorem, we have 
        \[ (1 + x)^{n} = \sum_{k = 0}^{n} \binom{n}{k}x^{k}. \]
        Taking derivatives with respect to x both sides gives 
        \[ n(1 + x)^{n - 1} = \sum_{k = 0}^{n} k \binom{n}{k}x^{k-1}.\]
        Plugging in $x = 1$ gives the result.

        \item \begin{align*}
            \sum_{k = 1}^{n} (-1)^{k + 1} k\binom{n}{k}
            &= \sum_{k = 1}^{n} (-1)^{k + 1} n\binom{n - 1}{k - 1} \\
            &= n\sum_{j = 0}^{n} (-1)^{j} \binom{n - 1}{j} \\
            &= 0 \quad (\text{From part a.})
        \end{align*}
        Here we used the formula $k\binom{n}{k} = n\binom{n - 1}{k - 1}, k > 0$.
    \end{enumerate}

    % 1.28
    \item 

    % 1.29
    \item 
\end{enumerate}

\end{document}