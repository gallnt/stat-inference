\documentclass{article}
\usepackage{amsmath}
\usepackage{amssymb}
\usepackage[margin=1.5in]{geometry}

\title{Statistical Inference Chapter 1}
\author{Gallant Tsao}

\begin{document}

\maketitle

\begin{enumerate}
    % 1.1
    \item \begin{enumerate}
        \item $\Omega = \{(x_1, x_2, x_3, x_4): x_i \in \{H, T\}\}.$
        \item If there are N leaves on the plant, $\Omega = [N]$.
        \item $\Omega = \{t: t \in \mathbb{R},\ t \geq 0\}$.
        \item $\Omega = \{w: w \in \mathbb{R}_{+}\}$.
        \item If there are n components, $\Omega = \{i/n: i \in \{0, 1, ..., n\}\}$.
    \end{enumerate}

    % 1.2
    \item \begin{enumerate}
        \item \begin{align*}
            x \in A \setminus B 
            &\iff x \in A \text{ and } x \notin B \\
            &\iff x \in A \text{ and } x \notin A \cap B \\
            &\iff x \in A \setminus (A \cap B).
        \end{align*}
        Also, 
        \begin{align*}
            x \in A \setminus B
            &\iff x \in A \text{ and } x \notin B \\
            &\iff x \in A \text{ and } x \in B^{c} \\
            &\iff x \in A \cap B^{c}.
        \end{align*}
        Therefore $A \setminus B = A \setminus (A \cap B) = A \cap B^{c}$.

        \item By the distributive law, \begin{align*}
            (B \cap A) \cup (B \cap A^{c}) 
            &= B \cap (A \cup A^c) \\
            &= B.
        \end{align*}

        \item \begin{align*}
            x \in B \setminus A
            &\iff x \in B \text{ and } x \notin A \\
            &\iff x \in B \text{ and } x \in A^c \\
            &\iff x \in B \cap A^c.
        \end{align*}

        \item From part b), we have \begin{align*}
            A \cup B
            &= A \cup ((B \cap A) \cup (B \cap A^c)) \\
            &= A \cup (B \cap A) \cup A \cup (B \cap A^c) \\
            &= A \cup A \cup (B \cap A^c) \\
            &= A \cup (B \cap A^c).
        \end{align*}
    \end{enumerate}

    % 1.3
    \item \begin{enumerate}
        \item \begin{align*}
            x \in A \cup B 
            &\iff x \in A \text{ or } x \in B \\
            &\iff x \in B \cup A. \\
            x \in A \cap B
            &\iff x \in A \text{ and } x \in B \\
            &\iff x in B \cap A.
        \end{align*}

        \item \begin{align*}
            x \in A \cup (B \cup C) 
            &= x \in A \text{ or } x \in B \cup C \\
            &= x \in A \cup B \text{ or } x \in C \\
            &= x \in (A \cup B) \cup C.
        \end{align*}

        \item \begin{align*}
            x \in (A \cup B)^{c}
            &\iff x \notin A \cup B \\
            &\iff x \in A^c \text{ and } x \in B^c \\
            &\iff x \in A^c \cap B^c. \\
            x \in (A \cap B)^c
            &\iff x \notin A \cap B \\
            &\iff x \in A^c \text{ or } x \in B^c \\
            &\iff x \in A^c \cup B^c.
        \end{align*}
    \end{enumerate}

    % 1.4
    \item \begin{enumerate}
        \item This is $P(A \cup B)$, so we get $P(A) + P(B) - P(A \cap B)$.
        \item This is $P(A \Delta B)$, so we get $P(A) + P(B) - 2P(A \cap B)$.
        \item This is again $P(A \cup B)$, so we get $P(A) + P(B) - P(A \cap B)$.
        \item This is $P((A \cap B)^c)$, so we get $1 - P(A \cap B)$.
    \end{enumerate}

    % 1.5
    \item \begin{enumerate}
        \item $A \cap B \cap C = \{ \text{a U.S. birth resulting in identtical twin females} \}$.
        \item $P(A \cap B \cap C) = \frac{1}{90} \cdot \frac{1}{3} \cdot \frac{1}{2} 
        = \frac{1}{540}$.
    \end{enumerate}

    % 1.6
    \item $p_0 = (1 - u)(1 - w), p_1 = u(1 - w) + w(1 - u), p_2 = uw$. For them to be equal, 
    \begin{align*}
        p_0 = p_2 
        &\implies 1 - u - w + uw = uw \\
        &\implies u + w = 1, \\
        p_1 = p_2
        &\implies u + w - 2uw = uw \\
        &\implies uw = \frac{1}{3}.
    \end{align*}
    The above two equations imply $u(1 - u) = \frac{1}{3}$, which has no real solutions in 
    $\mathbb{R}$. Therefore we can't choose such $u, w$ satisfying $p_0 = p_1 = p_2$.
    % 1.7
    \item \begin{enumerate}
        \item This is just having an extra case of hitting outside of the dart board. So 
        \[ P(\text{scoring } i \text{ points}) = \begin{cases}
            1 - \frac{\pi r^2}{A} & i = 0 \\
            \frac{\pi r^2}{A} \cdot \frac{1}{5^2} ((6 - i)^2 - (5 - i)^2) & i = 1, ..., 5
        \end{cases} \]

        \item \begin{align*}
            P(\text{scoring } i \text{ points}|\text{board is hit})
            &= \frac{P(\text{scoring } i \text{ points, board is hit})}
            {P(\text{board is hit})} \\
            &= \frac{\pi r^2}{A} \cdot \frac{1}{5^2} ((6 - i)^2 - (5 - i)^2) 
            / \frac{\pi r^2}{A} \\
            &= \frac{1}{5^2} ((6 - i)^2 - (5 - i)^2), \ i = 1, ..., 5
        \end{align*}
        For $i = 0$, we will definitely score given that we hit the board so \\
        $P(\text{scoring 0 points} | \text{board is hit}) = 0$, which is consistent with the 
        probability distribution in Example 1.2.7 as well.
    \end{enumerate}

    % 1.8
    \item \begin{enumerate}
        \item From the example given, \[
            P(\text{scoring } i \text{ points}) = \frac{(6 - i)^2 - (5 - i)^2}{5^2}, i = 1, ..., 5.
            \]

        \item Expanding the above, 
        \[ \frac{(6 - i)^2 - (5 - i)^2}{5^2} = \frac{11 - 2i}{r^2}, \]
        which is a decreasing function of $i$.

        \item \[
        \frac{11 - 2i}{5^2} > 0 \text{for } i = 1, ..., 5
        \]
        hence the first axiom is satisfied.
        \[ P(S) = P(\text{hitting the board}) = 1, \]
        hence the second axiom is satisfied. For $i \neq j$,
        \[ P(i \cup j) = \text{Area of ring } i + \text{Area of ring } j 
        = P(i) + P(j), \]
        hence the third axiom is satisfied so $P(\text{scoring } i \text{ points})$ 
        is a probability function.
    \end{enumerate}
    
    % 1.9
    \begin{enumerate}
        \item Suppose $x \in (\cup_{\alpha} A_{\alpha})^c$. Then $x \notin A_{\alpha}$ for all 
        $\alpha \in \Gamma$ so $x \in A_{\alpha}^{c}$ for all $\alpha \in \Gamma$. Therefore 
        $x \in \cap_{\alpha} A_{\alpha}$.

        Now suppose $x \in \cap_{\alpha} A_{\alpha}^{c}$. Then for all $\alpha \in \Gamma$, 
        $x \in A_{\alpha}^{c}$ hence $x \notin A_{\alpha}$, then 
        $x \notin \cup_{\alpha} A_{\alpha}$ so $x \in (\cup_{\alpha} A_{\alpha})^{c}$.

        \item Suppose $x \in (\cap_{\alpha} A_{\alpha})^{c}$. Then 
        $x \notin \cap_{\alpha} A_{\alpha}$ so $x \notin A_{\alpha}$ for some 
        $\alpha \in \Gamma$. Then $x \in A_{\alpha}^{c}$ for some $\alpha \in \Gamma$. 
        Therefore $x \in \cup_{\alpha} A_{\alpha}^{c}$.

        Now suppose $x \in \cup_{\alpha} A_{\alpha}^{c}$. Then $x \in A_{\alpha}^{c}$ for some 
        $\alpha \in \Gamma$ so $x \notin A_{\alpha}$ for some $\alpha \in \Gamma$. Then 
        $x \notin \cap_{\alpha}$ thus $x \in (\cap_{\alpha})^{c}$.
    \end{enumerate}

    % 1.10
\end{enumerate}

\end{document}