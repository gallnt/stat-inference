\documentclass{article}
\usepackage{amssymb}
\usepackage{amsmath}
\usepackage[margin=1in]{geometry}

\DeclareMathOperator{\var}{Var}
\DeclareMathOperator{\corr}{Corr}

\title{Statistical Inference Chapter 5}
\author{Gallant Tsao}

\begin{document}

\maketitle

\begin{enumerate}
    % 5.1
    \item Let $X$ be the number of color blinded people in that population of size $n$. Then 
    $X \sim \text{Binomial}(n, 0.01)$. We want 
    \begin{align*}
        P(X \geq 1) 
        &= 1 - P(X = 0) \\
        &= 1 - \binom{n}{0}(0.01)^{0}(0.99)^{n} \\
        &= 1 - 0.99^{n} \\
        &> 0.95.
    \end{align*}
    For the inequality above to be true, we need $n > \log_{0.99}(0.05) \implies n \geq 299$.

    % 5.2
    \item \begin{enumerate}
        \item 
    \end{enumerate}

    % 5.3
    \item From the definition, we can see that $Y_i \sim \text{Bernoulli}(1 - F(\mu))$. 
    Since 
    $\sum_{i = 1}^{n} Y_{i}$ is a sum of independent Bernoulli random variables, we can get 
    that \[ \sum_{i = 1}^{n} Y_{i} \sim \text{Binomial}(n, 1 - F(\mu)). \]

    % 5.4
    \item \begin{enumerate}
        \item 
    \end{enumerate}

    % 5.5
    \item Let $Y = \sum_{i = 1}^{n} X_i$. We can see that 
    $\bar{X} = (1/n)Y$ is a scale transformation. Then the pdf of $\bar{X}$ is 
    \[ f_{\bar{X}}(x) = \frac{1}{1/n} f_{Y}(\frac{x}{1/n}) = nf_{X}(nx). \]

    % 5.6
    \item \begin{enumerate}
        \item Set $Z = X - Y$, $W = X$. Then the Jacobian of $(X, Y)$ to $(Z, W)$ is 
    \end{enumerate}

    % 5.7
    \item \begin{enumerate}
        \item 
    \end{enumerate}

    % 5.8
    \item 5.8
    
    % 5.9
    \item 5.9

    % 5.10
    \item 5.10
    
    % 5.11
    \item 5.11
    
    % 5.12
    \item First note that since $X_1, \cdots, X_n$ are $N(0, 1)$, $\bar{X} \sim N(0, \frac{1}{n})$. In 
    particular, $Y_1 = |\bar{x}|$ is a folded normal distribution with mean 0 and variance $\frac{1}{n}$. 
    Then we can directly obtain that the expectation for $Y_1$ is 
    \[ \mathbb{E}[Y_1] = \sqrt{\frac{2}{\pi n}}. \]
    Similarly, we can see the $Y_2$ is the average of $n$ folded normals with mean 0 and variance 1, so 
    \[ \mathbb{E}[Y_2] = \frac{1}{n} \bigl( n\sqrt{\frac{2}{\pi}} \bigr) = \sqrt{\frac{2}{\pi}}. \]
    From the above we can see clearly that $\mathbb{E}[Y_1] \leq \mathbb{E}[Y_2]$.

    % 5.13
    \item 5.13
    
    % 5.14
    \item 5.14
    
    % 5.15
    \item 5.15
    
    % 5.16
    \item \begin{enumerate}
        \item \[ (X_1 - 1)^2 + \bigl( \frac{X_2 - 2}{2} \bigr)^2 + \bigl( \frac{X_3 - 3}{3} \bigr)^2 
        \sim \chi_3^2. \]

        \item \[ \frac{X_1 - 1}{\sqrt{\bigl( \frac{X_2 - 2}{2} \bigr)^2 + \bigl( \frac{X_3 - 3}{3} \bigr)^2}
        \big/ 2} \sim t_2. \]

        \item Since $F_{1, 2} \sim T_2^2$, squaring the random variable from part (b) gives the result.
    \end{enumerate}
    
    % 5.17
    \item \begin{enumerate}
        \item \[ f_X(x) = \frac{\Gamma(\frac{p + q}{2}) p^{p / 2} q^{q / 2} x^{p / 2 - 1}}
        {\Gamma(\frac{p}{2}) \Gamma(\frac{q}{2}) (q + px)^{(p + q) / 2}}, \ x > 0. \]

        \item We can write $X = \frac{U / p}{V / q}, U \sim \chi_p^2, V \sim \chi_q^2$. Firstly, note that 
        for $Y \sim \chi_n^2$, 
        \begin{align*}
            \mathbb{E}[Y^k] 
            &= \int_{0}^{\infty} x^k \cdot \frac{x^{n / 2 - 1}e^{-x / 2}}{2^{n / 2} \Gamma(\frac{n}{2})} 
            \ dx \\
            &= \int_{0}^{\infty} \frac{x^{n / 2 + k - 1}e^{-x / 2}}{2^{n / 2} \Gamma(\frac{n}{2})} 
            \ dx \\
            &= \frac{\Gamma(\frac{n}{2} + k) 2^k}{\Gamma(\frac{n}{2})} \int_{0}^{\infty} 
            \frac{x^{n / 2 + k - 1}e^{-x / 2}}{2^{n / 2 + k} \Gamma(\frac{n}{2} + k)} \ dx \\
            &= \frac{\Gamma(\frac{n}{2} + k) 2^k}{\Gamma(\frac{n}{2})}, \ k > -\frac{n}{2}.
        \end{align*}
        Plugging in $k = -1$ into $V$ gives $\mathbb{E}[V^{-1}] = \frac{1}{q - 2}$. Then 
        \begin{align*}
            E[X] 
            &= \mathbb{E} \bigl[ \frac{U / p}{V / q} \bigr] \\
            &= \frac{1}{pq} \cdot \mathbb{E}[U]\mathbb{E}[V] \\
            &= \frac{q}{q - 2}, \ q > 2.
        \end{align*}

        As for the variance, first note that 
        \[ \mathbb{E}[X^2] = \frac{q^2}{p^2} \mathbb{E}[U^2]\mathbb{E}[V^{-2}]. \]
        From the equation above, plugging in $k = 2$ and $k = -2$ repectively gives 
        \[ \mathbb{E}[U^2]= \frac{4\Gamma(\frac{p}{2} + 2)}{\Gamma(\frac{p}{2} + 2)} 
        = 4\bigl( \frac{p}{2} + 1 \bigr) \frac{p}{2} = 2p + p^2, \]
        \[ \mathbb{E}[V^{-2}] = \frac{2^{-2} \Gamma(\frac{q}{2} - 2)}{\Gamma(\frac{q}{2})} 
        = \frac{1}{4(\frac{q}{2} - 1)(\frac{q}{2} - 2)} = \frac{1}{(q - 2)(q - 4)}, \ q > 4. \]
        Finally we get that 
        \begin{align*}
            \var{X} 
            &= \mathbb{E}[X^2] - (\mathbb{E}[X])^2 \\
            &= \frac{q^2(2p + p^2)}{p^2(q - 2)(q - 4)} - \frac{q^2}{(q - 2)^2} \\
            &= \frac{2q^2(p + q - 2)}{p(q - 2)^2 (q - 4)}, \ q > 4.
        \end{align*}

        \item Let $U$ and $V$ be as given above. Then 
        \[ \frac{1}{X} = \frac{V / q}{U / p} \sim F_{q, p}. \]

        \item 
    \end{enumerate}
    
    % 5.18
    \item First note that if $X \sim t_p$, $X \sim \frac{Z}{\sqrt{V / p}}, \ V \sim \chi_p^2$, with $Z$ and 
    $V$ independent. The moments for $V$ can be obtained from the equation in part (b) of Exercise 5.17.
    \begin{enumerate}
        \item \[ \mathbb{E}[X] = \sqrt{p}\mathbb{E}[Z]\mathbb{E}[\frac{1}{\sqrt{V}}] = 0. \]
        \[ \var{X} = \mathbb{E}[X^2] = \mathbb{E}[\frac{Z^2}{V / p}] 
        = p \mathbb{E}[Z^2]\mathbb{E}[V^{-1}] = \frac{p}{p - 2}, \ p > 2. \]

        \item \[ X^2 \sim \frac{Z^2}{V / p} \sim \frac{\chi_1^2 / 1}{\chi_p^2 / p} \sim F_{1, p}. \]

        \item 
    \end{enumerate}
    
    % 5.19
    \item \begin{enumerate}
        \item $\chi_p^2 \sim \chi_q^2 + \chi_d^2$ where $\chi_q^2, \chi_d^2$ are independent random 
        variables and $d = p - q$. Since $\chi_d^2$ is a strictly positive random variable, for all $a > 0$, 
        \[ P(\chi_p^2 > a) = P(\chi_q^2 + \chi_d^2 > a) > P(\chi_q^2 > a). \]

        \item 
    \end{enumerate}
    
    % 5.20
    \item 5.20
    
    % 5.20
    \item 5.21
    
    
\end{enumerate}

\end{document}