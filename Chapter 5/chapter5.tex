\documentclass{article}
\usepackage{amssymb}
\usepackage{amsmath}
\usepackage[margin=1.5in]{geometry}

\DeclareMathOperator{\var}{Var}
\DeclareMathOperator{\corr}{Corr}

\title{Statistical Inference Chapter 5}
\author{Gallant Tsao}

\begin{document}

\maketitle

\begin{enumerate}
    % 5.1
    \item Let $X$ be the number of color blinded people in that population of size $n$. Then 
    $X \sim \text{Binomial}(n, 0.01)$. We want 
    \begin{align*}
        P(X \geq 1) 
        &= 1 - P(X = 0) \\
        &= 1 - \binom{n}{0}(0.01)^{0}(0.99)^{n} \\
        &= 1 - 0.99^{n} \\
        &> 0.95.
    \end{align*}
    For the inequality above to be true, we need $n > \log_{0.99}(0.05) \implies n \geq 299$.

    % 5.2
    \item \begin{enumerate}
        \item 
    \end{enumerate}

    % 5.3
    \item From the definition, we can see that $Y_i \sim \text{Bernoulli}(1 - F(\mu))$. 
    Since 
    $\sum_{i = 1}^{n} Y_{i}$ is a sum of independent Bernoulli random variables, we can get 
    that \[ \sum_{i = 1}^{n} Y_{i} \sim \text{Binomial}(n, 1 - F(\mu)). \]

    % 5.4
    \item \begin{enumerate}
        \item 
    \end{enumerate}

    % 5.5
    \item Let $Y = \sum_{i = 1}^{n} X_i$. We can see that 
    $\bar{X} = (1/n)Y$ is a scale transformation. Then the pdf of $\bar{X}$ is 
    \[ f_{\bar{X}}(x) = \frac{1}{1/n} f_{Y}(\frac{x}{1/n}) = nf_{X}(nx). \]

    % 5.6
    \item \begin{enumerate}
        \item 
    \end{enumerate}

    % 5.7
\end{enumerate}

\end{document}