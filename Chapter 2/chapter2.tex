\documentclass{article}
\usepackage{amssymb}
\usepackage{amsmath}

\title{Statistical Inference Chapter 2}
\author{Gallant Tsao}

\begin{document}

\maketitle

\begin{enumerate}
    % 2.1
    \item
    \begin{enumerate}
        % 1a
        \item Let $g(x)=x^3$. Then $g$ is monotonically increasing on $(0, 1)$. We get
        \[ 
        g^{-1}(y) = y^{1/3} \implies \frac{d}{dy}g^{-1}(y) = \frac{1}{3y^{2/3}}. 
        \]
        Since $X \in (0, 1), \ Y = X^3 \in (0, 1)$. Then by Theorem 2.1.5, 
        \begin{align*}
        f_{Y}(y) 
            &= f_{X}(g^{-1}(y)) \Bigl|\frac{d}{dy}g^{-1}(y)\Bigr| \\
            &= 42(y^{1/3})^{5}(1 - y^{1/3}) \cdot \frac{1}{3y^{2/3}} \\
            &= 14y(1 - y^{1/3}), \ y \in (0, 1).
        \end{align*}
        We also have 
        \begin{align*}
            \int_{0}^{1} 14y(1 - y^{1/3}) \ dy 
            &= 14\int_{0}^{1} y - y^{4/3} \ dy \\
            &= 14 \Big[ \frac{1}{2}y^2 - \frac{3}{7}y^{7/3} \Big]_{0}^{1} \\
            &= 14 ( \frac{1}{2} - \frac{3}{7}) \\
            &= 1.
        \end{align*}

        % 1b
        \item Let $g(x) = 4x + 3$. Then $g$ is monotonically increasing on $(0, \infty)$. We get 
        \[
        g^{-1}(y) = \frac{y - 3}{4} \implies \frac{d}{dy}g^{-1}(y) = \frac{1}{4}.
        \]
        Since $X \in  (0, \infty), \ Y = 4X + 3 \in (3, \infty)$. Then by Theorem 2.1.5,
        \begin{align*}
            f_{Y}(y)
            &= f_{X}(g^{-1}(y)) \Bigl|\frac{d}{dy}g^{-1}(y)\Bigr| \\
            &= 7e^{-7 \cdot \frac{y - 3}{4}} \cdot \frac{1}{4} \\
            &= \frac{7}{4}e^{\frac{21}{4} - \frac{7}{4}y}, \ y \in (3, \infty).
        \end{align*}
        We also have 
        \begin{align*}
            \int_{3}^{\infty} \frac{7}{4}e^{\frac{21}{4} - \frac{7}{4}y} \ dy 
            &= \frac{7}{4}e^{\frac{21}{4}} \int_{3}^{\infty} e^{-\frac{7}{4}y} \ dy \\
            &= \frac{7}{4}e^{\frac{21}{4}} \Big[ -\frac{4}{7}e^{-\frac{7}{4}y} \Big]_{3}^{\infty} \\
            &= \frac{7}{4}e^{\frac{21}{4}} (\frac{4}{7}e^{-\frac{21}{4}}) \\
            &= 1.
        \end{align*}

        % 1c
        \item Let $g(x) = x^2$. Then $g$ is monotonically increasing on $(0, 1)$. We get 
        \[
        g^{-1}(y) = \sqrt{y} \implies \frac{d}{dy}g^{-1}(y) = \frac{1}{2\sqrt{y}}.
        \]
        Since $X \in (0, 1), \ Y = X^2 \in (0, 1)$. Then by Theorem 2.1.5, 
        \begin{align*}
            f_{Y}(y) 
            &= f_{X}(g^{-1}(y)) \Bigl|\frac{d}{dy}g^{-1}(y)\Bigr| \\
            &= 30y(1 - \sqrt{y})^2 \cdot \frac{1}{2\sqrt{y}} \\
            &= 15\sqrt{y}(1 - \sqrt{y})^2, \ y \in (0, 1).
        \end{align*}
        We also have 
        \begin{align*}
            \int_{0}^{1} 15\sqrt{y}(1 - \sqrt{y})^2 \ dy
            &= 15 \int_{0}^{1} \sqrt{y} - 2y + y^{3/2} \ dy \\
            &= 15 \Big[ \frac{2}{3}y^{3/2} - y^2 + \frac{2}{5}y^{5/2} \Big]_{0}^{1} \\
            &= 15(\frac{2}{3} - 1 + \frac{2}{5}) \\
            &= 1.
        \end{align*}
        
    \end{enumerate}

    % 2.2
    \item 
    \begin{enumerate}
        \item Let $g(x) = x^2$. Then $g$ is monotonically increasing on $(0, 1)$. We get 
        \[
        g^{-1}(y) = \sqrt{y} \implies \frac{d}{dy}g^{-1}(y) = \frac{1}{2\sqrt{y}}.
        \]
        Since $X \in (0, 1), \ Y = X^2 \in (0, 1)$. Then by Theorem 2.1.5,
        \begin{align*}
            f_{Y}(y)
            &= f_{X}(g^{-1}(y)) \Bigl|\frac{d}{dy}g^{-1}(y)\Bigr| \\
            &= 1 \cdot \frac{1}{2\sqrt{y}} \\
            &= \frac{1}{2\sqrt{y}}, \ y \in (0, 1).
        \end{align*}

        \item Let $g(x) = -\log{x}$. Then $g$ is monotonically decreasing on $(0, 1)$. We get 
        \[
        g^{-1}(y) = e^{-y} \implies \frac{d}{dy}g^{-1}(y) = -e^{-y}.
        \]
        Since $X \in (0, 1), \ Y = \log{X} \in (0, \infty)$. Then by Theorem 2.1.5, 
        \begin{align*}
            f_{Y}(y)
            &= f_{X}(g^{-1}(y)) \Bigl|\frac{d}{dy}g^{-1}(y)\Bigr| \\
            &= \frac{(n + m + 1)!}{n!m!}e^{-ny}(1 - e^{-y})^{m} \cdot |-e^{-y}| \\
            &=  \frac{(n + m + 1)!}{n!m!}e^{-y(n + 1)}(1 - e^{-y})^{m}, \ y \in (0, \infty).
        \end{align*}

        \item Let $g(x) = e^{x}$. Then $g$ is monotonically increasing on $(0, \infty)$. We get 
        \[
        g^{-1}(y) = \ln{y} \implies \frac{d}{dy}g^{-1}(y) = \frac{1}{y}.
        \]
        Since $X \in (0, \infty), \ Y = e^{X} \in (0, \infty)$. Then by Theorem 2.1.5, 
        \begin{align*}
            f_{Y}(y)
            &= f_{X}(g^{-1}(y)) \Bigl|\frac{d}{dy}g^{-1}(y)\Bigr| \\
            &= \frac{1}{\sigma^2} \ln{y} e^{-(\ln{y}/\sigma)^2 /2} \cdot \frac{1}{y} \\
            &= \frac{1}{\sigma^2} \frac{\ln{y}}{y} e^{-(\ln{y}/\sigma)^2 / 2}, \ y \in (0, \infty).
        \end{align*}
    \end{enumerate}

    % 2.3
    \item First of all, 
    \[ X \in \{0, 1, 2, ...\} \implies Y \in \Big\{0, \frac{1}{2}, \frac{2}{3}, ...\Big\}. \]
    Then
    \begin{align*}
        P(Y = y) 
        &= P(\frac{X}{X + 1} = y) \\
        &= P(1 - \frac{1}{X + 1} = y) \\
        &= P(X = \frac{y}{1 - y}) \\
        &= \frac{1}{3} \Big( \frac{2}{3} \Big)^{y / (1 - y)}, \ y \in \Big\{\frac{k}{k + 1}: k \in \mathbb{N}_0 \Big\}.
    \end{align*}
    
    % 2.4
    \item \begin{enumerate}
        \item It is not hard to see that $f(x) \geq 0 \ \forall x \in \mathcal{X}$ as both piecewise functions are 
        exponentials. We also have 
        \begin{align*}
            \int_{-\infty}^{\infty} f(x) \ dx 
            &= \int_{-\infty}^{0} \frac{1}{2}\lambda e^{\lambda x} 
            + \int_{0}^{\infty} \frac{1}{2}\lambda e^{-\lambda x} \ dx \\
            &= \Big[\frac{1}{2}e^{\lambda x} \Big]_{-\infty}^{0} + \Big[ -\frac{1}{2}e^{-\lambda x} \Big]_{0}^{\infty} \\
            &= \frac{1}{2} + \frac{1}{2} \\
            &= 1.
        \end{align*}

        \item For $t \leq 0$, 
        \begin{align*}
            P(X < t) 
            &= \int_{-\infty}^{t} \frac{1}{2}\lambda e^{\lambda x} \ dx \\
            &= \Big[ \frac{1}{2}e^{\lambda x} \Big]_{\-\infty}^{t} \\
            &= \frac{1}{2}e^{\lambda t}.
        \end{align*}
        For $t > 0$,
        \begin{align*}
            P (X < t) 
            &= \frac{1}{2} + \int_{0}^{t} \frac{1}{2}\lambda e^{-\lambda x} \ dx \\
            &= \frac{1}{2} + \Big[ -\frac{1}{2}e^{-\lambda x} \Big]_{0}^{t} \\
            &= \frac{1}{2} + (-\frac{1}{2}e^{-\lambda t} + \frac{1}{2}) \\
            &= 1 - \frac{1}{2}e^{-\lambda t}.
        \end{align*}

        \item For $t \leq 0, \ P(|X| < t) = 0$. For $t > 0$,
        \begin{align*}
            P(|X| < t)
            &= P(-t < X < t) \\
            &= \int_{-t}^{0} \frac{1}{2}\lambda e^{\lambda x} \ dx 
            + \int_{0}^{t} \frac{1}{2}\lambda e^{-\lambda x} \ dx \\
            &= \Big[ \frac{1}{2}e^{\lambda x} \Big]_{-t}^{0} + \Big[ \frac{1}{2}e^{-\lambda x} \Big]_{0}^{t} \\
            &= \frac{1}{2} - \frac{1}{2}e^{-\lambda t} + (-\frac{1}{2}e^{-\lambda t} + \frac{1}{2}) \\
            &= 1 - e^{-\lambda t}.
        \end{align*}
    \end{enumerate}

    % 2.5
    \item Let $A_0 = \{\pi\}, A_1 = (0, \frac{\pi}{2}), A_2 = (\frac{\pi}{2}, \pi), A_3 = (\pi, \frac{3\pi}{2}), 
    A_4 = (\frac{3\pi}{2}, 2\pi)$, and let $g(x) = g_i(x) = \sin^2{x}$. Then for each $A_i (i \neq 0), g_i(x) = g(x) 
    \forall x \in A_i,$, $g_i(x)$ is monotone on $A_i$. Moreover, $\mathcal{Y} = (0, 1)$ is the same for all $i$, and 
    monotone on $A_i$, and 
    \[
    g^{-1}(y) = \arcsin{(\sqrt{x})} \implies \frac{d}{dy}g^{-1}(y) = \frac{1}{2\sqrt{y(1 - y)}}
    \]
    is continuous on $\mathcal{Y}$ for all $i$. Then by Theorem 2.1.8,
    \begin{align*}
        f_{Y}(y) 
        &= \sum_{i = 1}^{4} f_{X}(g^{-1}(y)) \Big| \frac{d}{dy}g_{i}^{-1}(y) \Big| \\
        &= 4 \cdot \frac{1}{2\pi} \cdot \Big| \frac{1}{2\sqrt{y(1 - y)}} \Big| \\
        &= \frac{1}{\pi\sqrt{y(1 - y)}}, \ y \in (0, 1).
    \end{align*}
    To use the cdf from (2.1.6), we first get that $x_1 = \arcsin{(\sqrt{y})}, x_2 = \pi - \arcsin{(\sqrt{y})}$. Note 
    \[ P(Y \leq y) = 2P(X \leq x_1) + 2P(X \leq \pi) - 2P(X \leq x_2) \]
    Then by differentiating the above we get 
    \begin{align*}
        f_{Y}(y)
        &= 2f_{X}(x_1) \cdot \frac{d}{dy}(\sin^{-1}{\sqrt{y}}) 
        - 2f_{X}(x_2) \cdot \frac{d}{dy}(\pi - \sin^{-1}{\sqrt{y}}) \\
        &= 2 \cdot \frac{1}{2\pi} \cdot \frac{1}{2\sqrt{y(1 - y)}} 
        - 2 \cdot \frac{1}{2\pi} \cdot (-\frac{1}{2\sqrt{y(1 - y)}}) \\
        &= \frac{1}{\pi\sqrt{y(1 - y)}}, \ y \in (0, 1).
    \end{align*}

    % 2.6
    \item \begin{enumerate}
        \item Let $g(x) = |x|^3, g_1(x) = -x^3, g_2(x) = x^3$. Let $A_0 = \{0\}, A_1 = (-\infty, 0), 
        A_2 = (0, \infty)$. Then we get $\mathcal{Y} = (0, \infty)$ so that all conditions for Theorem 2.1.8 are 
        satisfied. Then 
        \[ g_{1}^{-1}(y) = -y^{1/3} \implies \frac{d}{dy}g_{1}^{-1}(y) = -\frac{1}{3y^{2/3}}. \]
        \[ g_{2}^{-1}(y) = y^{1/3} \implies \frac{d}{dy}g_{2}^{-1}(y) = \frac{1}{3y^{2/3}}. \]        
        Then by Theorem 2.1.8,
        \begin{align*}
            f_{Y}(y) 
            &= \sum_{i = 1}^{2} f_{X}(g^{-1}(y)) \Big| \frac{d}{dy}g_{i}^{-1}(y) \Big| \\
            &= \frac{1}{2}e^{-y^{1/3}} \cdot \Big| -\frac{1}{3y^{2/3}} \Big|
            + \frac{1}{2}e^{-y^{1/3}} \cdot \Big| \frac{1}{3y^{2/3}} \Big| \\
            &= \frac{1}{3}y^{-2/3}e^{-y^{1/3}}, \ y \in (0, \infty).
        \end{align*}

        \item Let $g(x) = g_1(x) = g_2(x) = 1 - x^2$. Let $A_0 = \{0\}, A_1 = (-1, 0), A_2 = (0, 1)$. Then we get 
        \[ g_{1}^{-1}(y) = -\sqrt{1 - y} \implies \frac{d}{dy}g_{1}^{-1}(y) = \frac{1}{2\sqrt{1 - y}}, \]
        \[ g_{2}^{-1}(y) = \sqrt{1 - y} \implies \frac{d}{dy}g_{2}^{-1}(y) = -\frac{1}{2\sqrt{1 - y}}. \]
        Then we get $\mathcal{Y} = (0, 1)$ so that all conditions for Theorem 2.1.8 are satisfied. Then by 
        Theorem 2.1.8, 
        \begin{align*}
            f_{Y}(y)
            &= \sum_{i = 1}^{2} f_{X}(g^{-1}(y)) \Big| \frac{d}{dy}g_{i}^{-1}(y) \Big| \\
            &= \frac{3}{8}(-\sqrt{1 - y} + 1)^2 \cdot \Big| \frac{1}{2\sqrt{1 - y}} \Big| \\
            &\qquad \qquad + \frac{3}{8}(\sqrt{1 - y} + 1)^2 \cdot \Big| -\frac{1}{2\sqrt{1 - y}} \Big| \\
            &= \frac{3}{8} (1 - y - 2\sqrt{1 - y} + 1) \cdot \frac{1}{2\sqrt{1 - y}} \\
            &\qquad \qquad + \frac{3}{8} ( 1 - y + 2\sqrt{1 - y} + 1) \cdot \frac{1}{2\sqrt{1 - y}} \\
            &= \frac{3}{8}(1 - y)^{1/2} + \frac{3}{8}(1 - y)^{-1/2}, \ y \in (0, 1).
        \end{align*}
        (Note for $g_1$ we chose the negative root because $x < 0$).

        \item Let $g_1(x) = 1 - x^2, g_2(x) = 1 - x$. Let $A_0 = \{0\}, A_1 = (-1, 0), A_2 = (0, 1)$. Then we get 
        \[ g_{1}^{-1}(y) = -\sqrt{1 - y} \implies \frac{d}{dy}g_{1}^{-1}(y) = \frac{1}{2\sqrt{1 - y}}. \]
        \[ g_{2}^{-1}(y) = 1 - y \implies \frac{d}{dy}g_{2}^{-1}(y) = -1. \]
        Then we get $\mathcal{Y} = (0, 1)$ so that all conditions for Theorem 2.1.8 are satisfied. Then by 
        Theorem 2.1.8, 
        \begin{align*}
            f_{Y}(y)
            &= \sum_{i = 1}^{2} f_{X}(g^{-1}(y)) \Big| \frac{d}{dy}g_{i}^{-1}(y) \Big| \\
            &= \frac{3}{8}(-\sqrt{1 - y} + 1)^2 \cdot \Big| \frac{1}{2\sqrt{1 - y}} \Big| \\
            &\qquad \qquad + \frac{3}{8} (1 - y + 1)^2 \cdot |-1| \\
            &= \frac{3}{16\sqrt{1 - y}}(1 - \sqrt{1 - y})^2 + \frac{3}{8}(2 - y)^2, \ y \in (0, 1).
        \end{align*}
    \end{enumerate}

    % 2.7
    \item \begin{enumerate}
        \item For $g(x)=x^2, \ x \in [-1, 2]$, there is no partition $\{A_i\}$ of the interval which could produce the 
        same $\mathcal{Y}$ for all $i$. Therefore, we cannot use Theorem 2.1.8 in this case. To solve directly, we get 
        \begin{align*}
            f_{Y}(y) 
            &= \sum_{i = 1}^{4} f_{X}(g^{-1}(y)) \Big| \frac{d}{dy}g_{i}^{-1}(y) \Big| \\
            &= 
        \end{align*}
    \end{enumerate}

    % 2.8
    \item \begin{enumerate}
        \item It is easy to see that 
        \[ \lim_{x \to -\infty} F_{X}(x) = 0, \ \lim_{x \to +\infty} F_{X}(x) = 1. \]
        Moreover, both $0$ and $1-e^{-x}$ are non-decreasing on their respective intervals, and 
        \[ \lim_{x \to 0^+} F_{X}(x) = 0 \]
        so that $F_X$ is right continuous and therefore is a valud cdf. Its inverse is 
        \[ F_{X}^{-1}(y) = -\ln{(1 - y)}\]

        \item Again, we can see that 
        \[ \lim_{x \to -\infty} F_{X}(x) = 0, \ \lim_{x \to +\infty} F_{X}(x) = 1. \]
        $e^{x} / 2, 1 - (e^{-x} / 2)$ are increasing, and $1/2$ is noncreasing on their respective intervals, and 
        \[ \lim_{x \to 0} F_{X}(x) = \frac{1}{2}, \ \lim_{x \to 1} F_{X}(x) = \frac{1}{2} \]
        so that $F_X$ is continuous hence right continuous so is a valid cdf. Its inverse is
        \[ F_{X}^s{-1}(y) = \begin{cases}
            \ln{(2x)} & 0 < y < \frac{1}{2} \\
            -\ln{(2-2x)} & \frac{1}{2} \leq y < 1.
        \end{cases} \]

        \item Again, we can see that 
        \[ \lim_{x \to -\infty} F_{X}(x) = 0, \ \lim_{x \to +\infty} F_{X}(x) = 1. \]
        $e^{x}/4, 1 - (e^{-x} / 4)$ are both increasing on their respective intervals, and 
        \[ \lim_{x \to 0^+} F_{X}(x) = \frac{3}{4} = F_{X}(0) \]
        so that $F_X$ is right continuous and therefore is a valid cdf. Its inverse is 
        \[ F_{X}^{-1}(y) = \begin{cases}
            \ln{(4x)} & 0 < y < \frac{1}{4} \\
            -\ln{(4-4x)} & \frac{3}{4} \leq y < 1
        \end{cases}\]
    \end{enumerate}

    % 2.9
    We first find the cdf of $X$:
    \[ F_{X}(x) = \begin{cases}
        0 & x \leq 1 \\
        \frac{1}{4}(x - 1)^2 & 1 < x < 3 \\
        1 & x \geq 3
    \end{cases} \]
    Then we have 
    \[ \lim_{x \to 1} F_{X}(x) = 0, \ \lim_{x \to 3} F_{X}(x) = 1. \]
    hence $X$ has a continuous cdf. Let $u(x) = F_{X}(x)$. Then $u(x)$ is nondecreasing and by Theorem 2.1.10, 
    $Y = u(X)$ has a uniform distribution.

    % 2.10
    \begin{enumerate}
        \item 
    \end{enumerate}
\end{enumerate}

\end{document}